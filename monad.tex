\begin{frame}[fragile]
\frametitle{Monad}
\framesubtitle{The interface}
\begin{block}{Java 8/C\# with the addition of higher-kinded polymorphism}
\begin{lstlisting}[style=language,language=java]
interface Monad<T> {
  <A> T<A> join(T<T<A>> a);
  <X> T<X> unit(X x);
}
\end{lstlisting}
\end{block}
\begin{block}{Haskell}
\begin{lstlisting}[style=language,language=haskell]
class Monad t where
  join :: t (t a) -> t a
  unit :: x -> t x
\end{lstlisting}
\end{block}
\end{frame}

\begin{frame}[fragile]
\frametitle{Monad}
\begin{itemize}
\item The monad interface has laws too.
\item The monad interface has strictly stronger requirements than functor.
  \begin{itemize}
  \item In other words, all structures that are monads, are also functors.
  \item However, not all structures that are functors, are also monads.
  \end{itemize}
\item Therefore, there are fewer monad instances than functor instances.
\end{itemize}
\end{frame}

\begin{frame}[fragile]
\frametitle{Monad}
\framesubtitle{The instances}
\begin{block}{But still a \emph{very large} amount}
\begin{itemize}
\item \lstinline{List}
\item Reader \lstinline{((->) e)}
\item \lstinline{State s}
\item \lstinline{Continuation r}
\item \lstinline{Maybe/Nullable}
\item \lstinline{Exception}
\item \lstinline{Writer w}
\item \lstinline{Free f}
\end{itemize}
\end{block}
\end{frame}

\begin{frame}[fragile]
\frametitle{Monad}
\framesubtitle{The operations}
\begin{block}{and lots of operations too}
\begin{itemize}
\item \lstinline{sequence :: [t a] -> t [a]}
\item \lstinline{filterM :: (a -> t Bool) -> [a] -> t [a]}
\item \lstinline{findM :: (a -> t Bool) -> [a] -> Maybe [a]}
\end{itemize}
\end{block}
\end{frame}

\begin{frame}[fragile]
\frametitle{Monad}
\framesubtitle{Some mythbusting}
\begin{center}
\textbf{This} is what monad is for.
\end{center}
\begin{center}
\begin{itemize}
\item A lawful interface.
\item Satisfied by lots of instances.
\item Gives rise to lots of useful operations.
\end{itemize}
\end{center}
\end{frame}

\begin{frame}[fragile]
\frametitle{Monad}
\framesubtitle{Some mythbusting}
\begin{block}{Monad}
\begin{itemize}
\item<1-> \sout{for controlling side-effects.}
\item<2-> \sout{make my program impure.}
\item<3-> \sout{\emph{something blah something} IO.}
\item<4-> \sout{\emph{blah blah} in \lstinline{$SPECIFIC_PROGRAMMING_LANGUAGE}.}
\item<5-> \sout{\emph{blah blah} relating to \lstinline{$SPECIFIC_MONAD_INSTANCE}.}
\item<6-> \sout{\emph{Monads Might Not Matter, so use Actors instead}}\footnote{yes, seriously, this is a thing.}
\item<7-> Too much bullshizzles to continue enumerating.
\end{itemize}
\end{block}
\end{frame}
