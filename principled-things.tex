\begin{frame}
\frametitle{Principled Things}
\begin{itemize}
\item<1-> What do we mean by a principled thing?
\item<2-> Principled reasoning gives rise to useful inferences.
\begin{center}
\begin{tabular}{c@{\,}l@{}} 
  & $p$ \\
  & $p \to q$ \\\cline{2-2}
  $\therefore$         
  & $q$ \\
\end{tabular}
\end{center}
\end{itemize}
\end{frame}

\begin{frame}[fragile]
\frametitle{Principled reasoning is already familiar}
\framesubtitle{using Java/C\# syntax}
\begin{lstlisting}[style=language,language=java]
enum Order { LT, EQ, GT }

interface Compare<A> {
  Order compare(A a1, A a2);
}
\end{lstlisting}
\begin{block}{We define this interface because}
\begin{itemize}
\item We can produce data structures to satisfy the interface.
\item We can define operations that function on all instances of the interface.
\end{itemize}
\end{block}
\end{frame}

\begin{frame}
\frametitle{Principled Reasoning}
\begin{block}{Data structures such as}
\begin{itemize}
\item integers
\item strings
\item list of elements where the elements can be compared
\end{itemize}
\end{block}
\end{frame}

\begin{frame}
\frametitle{Principled Reasoning}
\begin{block}{Operations such as}
\begin{itemize}
\item \lstinline{List\#sort}
\item \lstinline{Tree\#insert}
\item \lstinline{List\#maximum}
\end{itemize}
\end{block}
\end{frame}

\begin{frame}
\frametitle{Principled Things}
\framesubtitle{Laws}
We might also define constraints required of instances.
\begin{block}{For example}
\begin{itemize}
\item if \lstinline{compare(x, y) == LT} then \lstinline{compare(y, x) == GT}
\item if \lstinline{compare(x, y) == EQ} then \lstinline{compare(y, x) == EQ}
\item if \lstinline{compare(x, y) == GT} then \lstinline{compare(y, x) == LT}
\end{itemize}
\end{block}
We will call these \emph{laws}. Laws enable reasoning on abstract code.
\end{frame}

\begin{frame}
\frametitle{Summary}
\begin{itemize}
\item a principled interface
\item law-abiding instances
\item derived operations
\end{itemize}
\end{frame}

\begin{frame}
\frametitle{Principled Reasoning for Practical Application}
\begin{itemize}
\item We try to maximise instances and derived operations, however, these two objectives often trade against each other.
\item For example, all things that can \lstinline{compare} can also be tested for equality, but not always the other way around\footnote{such as complex numbers}.
\item Obtaining the best practical outcome requires careful application of \emph{principled reasoning}.
\end{itemize}
\end{frame}
